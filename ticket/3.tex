%!TEX root = ../quantum.tex
\subsubsection{\textcolor{red}{Сохранение вероятности уравнением Шредингера}}
\subsubsection{Собственные функции и собственные значения. Понятие представления}

Пусть $\hat{A}$ -- оператор, соотвествущий наблюдаемой (физической) величине. $\exists$ операторное уравнение $\hat{A}\Psi=a\Psi$, где $a$ - неизвестная комплексная величина. $\exists \psi_n$ -- решения операторного уравнения (собственные функции) и соотвествущие им  $a_n$ (собственные числа). 

Спектр $a_n$ может быть как непрерывным, так и дискретным, и отвечает единственно возможным исходам эксперимента, например уровни энергии $E_n$ дискретного спектра атома. 

\subsubsection{Разложение вф по собств. функциям какого-либо оператора}

$\psi_n$ -- волновая функция (состояние системы), соотвествущее $a_n$. $\{\psi_n\}$ составляет базис в Гильбертовом пространстве, и в силу этого любую волновую функцию можно разложить в обобщенный ряд Фурье для дискретного спектра (или интеграл, если спектр непрерывный): 
$$
\Psi=\sum\limits_n C_n\psi_n, \quad \text{или} \quad
\Psi=\int\limits C(a)\psi(a)da
$$

В этом ряде $|C_n|^2$ есть вероятность того, что квантовая система с волновой функцией $\Psi$ находится в состоянии $\psi_n$ (принцип суперпозиции). Аналогично в интеграле $|C(a)|^2$ есть плотность вероятности нахождния системы в $[\Psi(a),\Psi(a+da)]$.

Заметим, что здесь мы работали в $a$ - представлении: раскладывали вф по собств. функциям оператора $\hat{A}$ и полученная функция описывает вероятностный характер в пространстве $a$.

\textbf{Примечание(не для ответа). Что такое операторное уравнение, его собственные числа и функции}. Рассмотрим ДУ
$$
\frac{d}{dx}y=ay \quad\Longleftrightarrow\quad \hat{A}y=ay
$$
Здесь оператор $\hat{A}=\frac{d}{dx}$ -- оператор дифференцирования, $a$ -- собственные числа оператора, $y=e^{ax}$ -- собственные функции оператора. Здесь спектр $a$ непрерывен.

\subsubsection{Задача. Отнормировать/нарисовать $\psi(x)=\Theta(x+L/2)+\Theta(x-L/2)$}


Дана ненормированная волновая функция $\psi(x)$:
\begin{gather*}
\psi(x)=\psi(x)=\Theta\left(x+\frac{L}{2}\right)-\Theta\left(x-\frac{L}{2}\right)
\end{gather*}
Отнормируйте и нарисуйте график плотности вероятности величины $x$.

\textbf{Решение}. Предположим, что наша функция есть произведение нормированной $\psi_0$ на произвольную константу $\frac{1}{c}$, найдем ее из условия нормировки:

\begin{gather*}
\int_{-\infty}^{\infty}|\psi_0(x)|^2 dx=
c^2\int_{-\infty}^{\infty}|\psi(x)|^2 dx=1\\
c^2\int_{-\infty}^{\infty}|\psi(x)|^2 dx=
c^2 L \quad\Rightarrow\quad c=\frac{1}{\sqrt{L}}
\end{gather*}
Мы нашли нормированную функцию:

\begin{gather*}
\psi_0=\frac{1}{\sqrt{L}}\left(
	\Theta\left(x+\frac{L}{2}\right)-\Theta\left(x-\frac{L}{2}\right)
\right)
\end{gather*}

$|\psi_0|^2$ имеет вид прямоугольника, высота его $\frac{1}{L}$, ширина от $-L/2$ до $L/2$.