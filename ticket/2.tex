%!TEX root = ../quantum.tex

\subsubsection{Уравнение Шредингера. Вывод перехода к классике с появлением уравнения Гамильтона-Якоби}

Нестационарное уравнение Шредингера $i\hbar\frac{\partial \psi}{\partial t}=\hat{H}\psi$. Будем искать его решение в виде $\psi=Ae^{i\Theta}$:

$$
i\hbar \frac{\partial \psi}{\partial t}=i\hbar \frac{\partial A}{\partial t}e^{i\Theta}-\hbar A\frac{\partial \Theta}{\partial t}e^{i\Theta}
$$

Учтем, что 

$$\hat{H}=\frac{\hat{p}^2}{2m}+U(r)=\frac{(-i\hbar \nabla)^2}{2m}+U(r)=\frac{-\hbar^2}{2m}\Delta+U(r)$$

и

\begin{gather*}
\Delta\psi=\nabla^2\left[A\cdot e^{i\Theta}\right]=
\nabla\left[\nabla A\cdot e^{i\Theta}+iA\nabla\Theta \cdot e^{i\Theta}\right]=
\nabla\left[
	\nabla A\cdot e^{i\Theta}
	\right]+\nabla\left[
		iA\nabla\Theta \cdot e^{i\Theta}
	\right]=\\=
\left[
	\Delta A \cdot e^{i\Theta}+i\nabla A\nabla\Theta \cdot e^{i\Theta}\right]+
i\left[
	\nabla A\nabla \Theta+A\Delta\Theta+
	iA\nabla \Theta \nabla \Theta
\right]e^{i\Theta}=\\=
\left[
	\Delta A +i\nabla A\nabla\Theta + i\nabla A\nabla \Theta+iA\Delta\Theta - A(\nabla \Theta)^2
\right]e^{i\Theta}=\\=
\left[
	\Delta A +2i\nabla A\nabla\Theta +iA\Delta\Theta - A(\nabla \Theta)^2
\right]e^{i\Theta}
% \Delta A\cdot e^{i\Theta} + A\cdot \Delta[e^{i\Theta}]=
% \Delta A\cdot e^{i\Theta} + A\cdot 
% (-e^{i\Theta})\left((\nabla \Theta)^2-i\Delta\Theta\right)
\end{gather*}
Тогда нестационарное уравнение Шредингера принимает вид:
\begin{gather*}
	i\hbar \frac{\partial A}{\partial t}-\hbar A\frac{\partial \Theta}{\partial t}=-\frac{\hbar^2}{2m}\left[
	\Delta A +2i\nabla A\nabla\Theta +iA\Delta\Theta - A(\nabla \Theta)^2+A\cdot U(r)
\right]
\end{gather*}
Разделим в нем реальные и мнимые части:
\begin{gather*}
	i\hbar \frac{\partial A}{\partial t}=-\frac{\hbar^2}{2m}
	(2i\nabla A\nabla\Theta+iA\Delta\Theta)\\
	-\hbar A\frac{\partial \Theta}{\partial t}=-\frac{\hbar^2}{2m}\left(\Delta A - A(\nabla \Theta)^2\right)+A\cdot U(r)
\end{gather*}
Работаем в приближении $\Delta A \ll (\nabla\Theta)^2$:
\begin{gather*}
	-\hbar A\frac{\partial \Theta}{\partial t}=\frac{\hbar^2}{2m}\left(A(\nabla \Theta)^2\right)+A\cdot U(r)\quad \bigg|\,:A\\
	-\hbar \frac{\partial \Theta}{\partial t}=\frac{\hbar^2}{2m}(\nabla \Theta)^2+U(r)
\end{gather*}
Воспользуемся соотношениями де-Бройля $S=\hbar\Theta$, $\nabla S=\vec{p}$:
\begin{gather*}
	\frac{\partial \Theta}{\partial t}=\frac{1}{\hbar}\frac{\partial S}{\partial t}, \qquad
	\nabla\Theta=\nabla S\frac{1}{\hbar}=\frac{\vec{p}}{\hbar}\\
%
	-\frac{\partial S}{\partial t}=\frac{\hbar^2}{2m}\frac{p^2}{\hbar^2}+U(\vec{r})=\frac{p^2}{2m}+U(\vec{r})=H
\end{gather*}
Окончательно получаем уравнение Гамильтона-Якоби:
\begin{equation*}
	\frac{\partial S}{\partial t}+H=0
\end{equation*}

\subsubsection{{Оператор в конкретном представлении. Матрица оператора.}}

В случае, когда спектр оператора является дискретным, возникают матричные представления. Пусть
$$\hat H \ket{n} = E_n \ket{n}, ~ \ket{n}\definition\ket{\psi_n}, $$
где $n=0,1,2,\dots$- индекс состояний дискретного спектра. Из условия полноты
$$\sum_n \ket n \bra n = 1$$
следует, что произвольный вектор состояния $\ket{\psi}$
может быть разложен по собственным векторам $\ket{n}$ следующим образом:
$$\ket{\psi}=\sum\limits_n \ket{n} \braket{n}{\psi} $$
Волновая функция $\braket{n}{\psi}$ в данном n- представлении является столбцом
$$
\braket{n}{\psi}=\norm{\mqty{\braket{1}{\psi}\\
				\braket{2}{\psi}\\
				\braket{3}{\psi}\\
				\dots }}
$$
Операторы в n-представлении являются матрицами
\begin{gather*}
\mel{n}{\hat A}{\psi}=\sum\limits_{n'} \mel{n}{\hat A}{n'}\braket{n'}{\psi}=\\
\sum\limits_{n'} A_{nn'}\braket{n'}{\psi}=
\norm{
	\begin{matrix}
		A_{11} & A_{12} & \dots \\
		A_{21} & A_{22} & \dots \\
		\dots  & \dots	& \dots \\
	\end{matrix}
}
\norm{
	\begin{matrix}
		\braket{1}{\psi} \\
		\braket{2}{\psi} \\
		\dots
	\end{matrix}
}
\end{gather*}

Рассмотрим условие эрмитовости оператора $\hat F$. По определению,
$$\mel{n}{\hat F}{n'}=\braket{\hat F^+ n}{n'}=\mel{n'}{\hat F^+}{n}^*.$$
В матричных обозначениях это переписывается так:
$$ F_{nn}=\qty(F^+)^*_{nn'}, $$
или
$$ \qty(F^+)_{n'n}=\qty(F_{n'n})^*=\qty(F^T)^*_{nn'}.$$

Таким образом, оператор эрмитов, то есть $\hat F= \hat F^+$, то
$$F=(F^T)^*\definition F^+ .$$
Следовательно, эрмитовым операторам соответствуют эрмитовые матрицы.

Найдем результат последовательного действия операторов $\hat A$ и $\hat B$.

$$\mel{n}{\hat A \hat B}{n'}=\sum\limits_{n''}\mel{n}{\hat A}{n'' }\mel{n''}{\hat B}{n'} =\sum\limits_{n''}A_{nn''}B_{n''n'}$$

Таким образом, матрица оператор, равного произведению операторов $\hat A$ и $\hat B$, равна произведению матриц, соответствующих этим операторам.
\subsubsection{Задача. Отнормировать/нарисовать $\psi(x)=1/\cosh(\alpha(x-x_0))$}

Дана ненормированная волновая функция $\psi(x)$:
\begin{gather*}
\psi(x)=\frac{1}{\cosh(\alpha(x-x_0))}
\end{gather*}
Отнормируйте и нарисуйте график плотности вероятности величины $x$.

\textbf{Решение}. Предположим, что наша функция есть произведение нормированной $\psi_0$ на произвольную константу $c$, найдем ее из условия нормировки:

\begin{gather*}
\int_{-\infty}^{\infty}|\psi_0(x)|^2 dx=
c^2\int_{-\infty}^{\infty}|\psi(x)|^2 dx=1
\end{gather*}

\begin{gather*}
\int_{-\infty}^{\infty}|\psi(x)|^2 dx=
\int_{-\infty}^{\infty} \cosh^{-2}(\alpha(x-x_0)) dx=\\
\textcolor{violet}{\left[ y=\alpha(x-x_0), dx=\frac{dy}{\alpha}\right]}=\\=
\frac{1}{\alpha}\int_{-\infty}^{\infty} \cosh^{-2}y dy=
\frac{4}{\alpha}\int_{-\infty}^{\infty} \frac{1}{(e^y+e^{-y})^2}dy=\\
\textcolor{violet}{\left[ e^y=t, dy=\frac{dt}{t}, t(y=-\infty)=0, t(y=\infty)=\infty\right]}=\\=
%
\frac{4}{\alpha}\int_0^{\infty} \frac{1}{t(t+1/t)^2}dt=
\frac{4}{\alpha}\int_0^{\infty} \frac{t}{(t^2+1)^2}dt=
\frac{2}{\alpha}\int_0^{\infty} \frac{d[t^2]}{(t^2+1)^2}dt=
\frac{2}{\alpha}\int_0^{\infty} \frac{d[t^2+1]}{(t^2+1)^2}dt=\\=
-\frac{2}{\alpha}\frac{1}{(t^2+1)}\bigg|_{0}^{\infty}=
\frac{2}{\alpha}
\end{gather*}

Тогда  $c^2=\frac{\alpha}{2} \Rightarrow c=\sqrt\frac{\alpha}{2} $.

Мы нашли нормированную функцию:

\begin{gather*}
\psi_0=\sqrt\frac{\alpha}{2}\frac{1}{\cosh^2(\alpha(x-x_0))},\qquad
|\psi_0|^2=\frac{\alpha}{2}\frac{1}{\cosh(\alpha(x-x_0))}
\end{gather*}

% <!---->
Колокообразный вид функции $|\psi_0|^2$ легко нарисовать,  если учесть, что 1) она всегда положительна 2) имеет максимум там, где $\cosh$ имеет минимум (т.е. в точке $x=x_0$, тогда максимум равен $\frac{\alpha}{2}$).