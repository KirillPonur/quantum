%!TEX root = ../quantum.tex

\subsubsection{Уравнение Шредингера. Вывод перехода к классике с появлением уравнения Гамильтона-Якоби}

Нестационарное уравнение Шредингера $i\hbar\frac{\partial \psi}{\partial t}=\hat{H}\psi$. Будем искать его решение в виде $\psi=Ae^{i\Theta}$:

$$
i\hbar \frac{\partial \psi}{\partial t}=i\hbar \frac{\partial A}{\partial t}e^{i\Theta}-\hbar A\frac{\partial \Theta}{\partial t}e^{i\Theta}
$$

Учтем, что 

$$\hat{H}=\frac{\hat{p}^2}{2m}+U(r)=\frac{(-i\hbar \nabla)^2}{2m}+U(r)=\frac{-\hbar^2}{2m}\Delta+U(r)$$

и

\begin{gather*}
\Delta\psi=\nabla^2\left[A\cdot e^{i\Theta}\right]=
\nabla\left[\nabla A\cdot e^{i\Theta}+iA\nabla\Theta \cdot e^{i\Theta}\right]=
\nabla\left[
	\nabla A\cdot e^{i\Theta}
	\right]+\nabla\left[
		iA\nabla\Theta \cdot e^{i\Theta}
	\right]=\\=
\left[
	\Delta A \cdot e^{i\Theta}+i\nabla A\nabla\Theta \cdot e^{i\Theta}\right]+
i\left[
	\nabla A\nabla \Theta+A\Delta\Theta+
	iA\nabla \Theta \nabla \Theta
\right]e^{i\Theta}=\\=
\left[
	\Delta A +i\nabla A\nabla\Theta + i\nabla A\nabla \Theta+iA\Delta\Theta - A(\nabla \Theta)^2
\right]e^{i\Theta}=\\=
\left[
	\Delta A +2i\nabla A\nabla\Theta +iA\Delta\Theta - A(\nabla \Theta)^2
\right]e^{i\Theta}
% \Delta A\cdot e^{i\Theta} + A\cdot \Delta[e^{i\Theta}]=
% \Delta A\cdot e^{i\Theta} + A\cdot 
% (-e^{i\Theta})\left((\nabla \Theta)^2-i\Delta\Theta\right)
\end{gather*}
Тогда нестационарное уравнение Шредингера принимает вид:
\begin{gather*}
	i\hbar \frac{\partial A}{\partial t}-\hbar A\frac{\partial \Theta}{\partial t}=-\frac{\hbar^2}{2m}\left[
	\Delta A +2i\nabla A\nabla\Theta +iA\Delta\Theta - A(\nabla \Theta)^2+A\cdot U(r)
\right]
\end{gather*}
Разделим в нем реальные и мнимые части:
\begin{gather*}
	i\hbar \frac{\partial A}{\partial t}=-\frac{\hbar^2}{2m}
	(2i\nabla A\nabla\Theta+iA\Delta\Theta)\\
	-\hbar A\frac{\partial \Theta}{\partial t}=-\frac{\hbar^2}{2m}\left(\Delta A - A(\nabla \Theta)^2\right)+A\cdot U(r)
\end{gather*}
Работаем в приближении $\Delta A \ll (\nabla\Theta)^2$:
\begin{gather*}
	-\hbar A\frac{\partial \Theta}{\partial t}=\frac{\hbar^2}{2m}\left(A(\nabla \Theta)^2\right)+A\cdot U(r)\quad \bigg|\,:A\\
	-\hbar \frac{\partial \Theta}{\partial t}=\frac{\hbar^2}{2m}(\nabla \Theta)^2+U(r)
\end{gather*}
Воспользуемся соотношениями де-Бройля $S=\hbar\Theta$, $\nabla S=\vec{p}$:
\begin{gather*}
	\frac{\partial \Theta}{\partial t}=\frac{1}{\hbar}\frac{\partial S}{\partial t}, \qquad
	\nabla\Theta=\nabla S\frac{1}{\hbar}=\frac{\vec{p}}{\hbar}\\
%
	-\frac{\partial S}{\partial t}=\frac{\hbar^2}{2m}\frac{p^2}{\hbar^2}+U(\vec{r})=\frac{p^2}{2m}+U(\vec{r})=H
\end{gather*}
Окончательно получаем уравнение Гамильтона-Якоби:
\begin{equation*}
	\frac{\partial S}{\partial t}+H=0
\end{equation*}