%!TEX root = ../quantum.tex
\subsubsection{От каких переменных может зависеть волновая функция. Полный набор.}



Волновая функция должна полным образом описывать систему, а изменение волновой функции от времени полностью описывает эволюцию квантово-механической системы (Уравнение Шредингера). 

Существует несколько вариантов представления волновых функций (координатный, импульсный, энергетический и т.д). Соответственно, они могут быть описаны через разные переменные. 

Максимальная информация о системе (Полнота описания системы) определяет количество степеней свободы. Чтобы волновая функция полностью описывала систему, необходимо, чтобы количество ее переменных равнялось количеству степеней свободы. 

Переменными волновой функции могут быть собственные значения какого-то оператора. Одновременно являться переменными волновой функции могут быть лишь те величины, операторы которых коммутируют между собой. Соответственно, переменные волновой функции могут быть взяты даже из различных операторов, но главное, чтобы выполнялось условие коммутации. Сколько переменных необходимо для полного описания волновой функции? Количество переменных должно равняться количеству степеней свободы.

Дальше идет отсебятина: уверенности нет, но скорее всего это так. говорите на свой страх и риск. 

 Пример: для описания квантово-механической системы, состоящей из одного свободного электрона, на самом деле недостаточно знать координат или импульсов этой частицы, так как она обладает как минимум еще и спином., что дает ему еще как минимум одну степень свободы. Получаем, что у электрона степеней свободы как минимум 4 (x,y,z, проекции координат или импульсов +спин). 


\subsubsection{\textcolor{red} {Дайте определение среднего значения физической величины в каком-либо
состоянии.}}

\subsubsection{\textcolor{red} {Докажите теорему о полноте (разложении единицы)
$\delta(a-a')=\int\Psi_b^*(a)\Psi_b(a')\dd{b}$.
Запишите её в абстрактных Дираковских обозначениях. Какова должна быть
нормировка.} }