%!TEX root = ../quantum.tex
\subsubsection{Какие значения может принимать некоторая физическая величина $A$ и с какой вероятность?
}


Физическая величина в квантовой механике есть наблюдаемая величина. Каждой наблюдаемой величине соответствует свой оператор. Существует операторное уравнение на собственные функции и собственные числа наблюдаемой $A$ .

$$\hat{A}\Psi=a\Psi$$
Решениями такого операторного уравнения являются пары $a_n,\Psi_n$. 

Если решения такого операторного уравнения счётны, то такой случай является дискретным. В противном случае, когда собственные числа не являются счётными, то такой случай называется непрерывным.

По постулатам квантовой механики: при измерении наблюдаемой величины получаются только собственные значения оператора $\hat{A}$. Если система наблюдается в состоянии $\Psi_n$,  то в результате её измерения получаем $a_n$ с вероятностью 1.

Если система находится в суперпозиции состояния $\Psi=c_1\Psi_1+c_2\Psi_2+\dots+C_k\Psi_k$, то при измерении системы вероятность получить $a_1$ есть $|c_1|^2$.