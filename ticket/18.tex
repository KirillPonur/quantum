%!TEX root = ../quantum.tex
\subsubsection{Вопрос 1}

Докажите, что если операторы коммутируют, то они имеют общие собственные
функции.


$$\hat{M}\hat{F}-\hat{F}\hat{M}=0$$
Пусть $\Psi_n$ образуют полную систему собственных функций оператор $\hat{M}$, то есть
$\hat{M}\Psi_n=M_n\Psi_n$.

Подействуем оператором $\hat{F}$

$$\hat{M}\hat{F}=\hat{F}\hat{M}$$
$$\hat{M}\hat{F}\Psi_n=\hat{F}\hat{M}\Psi_n=\hat{F}M_n\Psi_m=M_n\hat{F}\Psi_n $$
$$\hat{M}(\hat{F}\Psi_n)=M_n(\hat{F}\Psi_n), \hat{F\Psi_n}-\text{собственная функция оператора }\hat{M}$$
Следовательно $\hat{F}\Psi_n=\Psi_n$. $\hat{F}\Psi_n$ отличается от собственной функции только на константу, пусть $\const=F_n$

Тогда $\hat{F}\Psi_n=F_n\Psi_n.$

\subsubsection{Вопрос 3}
Докажите, что оператор кинетической энергии эрмитов.

$$\hat{T}=-\frac{\hbar^2}{2m}\nabla^2- \text{оператор кинетической энергии}$$
$$\hat{T_x}=-\frac{\hbar^2}{2m}\pdv[2]{x}$$
$$\big  \inftyint\phi^*\hat{T}\Psi \dd{x}=
\inftyint \hat{T}^+\phi^*\Psi \dd{x} - \text{определение эрмитовости}$$
\begin{gather*}  
\inftyint\phi^* \qty(\frac{-\hbar^2}{2m})
\qty(\pdv[2]{x}\Psi)\dd{x}=
\begin{bmatrix}
U=\phi^* & V=\pdv{\Psi}{x} \\
\dd{U}=\pdv{\phi^*}{x} & \dd{V}=\pdv[2]{x}\Psi\dd{x}
\end{bmatrix}=\\
\phi^*\pdv{\Psi}{x}\eval_{-\infty}^{\infty}\qty(\frac{\hbar^2}{2m})-
\qty(-\frac{\hbar^2}{2m})\inftyint
\pdv{\Psi}{x}\pdv{\phi^*}{x}\dd{x}
=*
\end{gather*}
В силу физических соображений первое слагаемое дает 0 (на бесконечности функция не может расти, а только спадать).

\begin{gather*}
*= -\qty(-\frac{\hbar^2}{2m})\inftyint
\pdv{\Psi}{x}\pdv{\phi^*}{x}\dd{x}= 
	\begin{bmatrix}
	U=\pdv{\phi^*}{x} & V=\Psi \\
	\dd{U}=\pdv[2]{\phi^*}{x}\dd{x} & \dd{V}=\pdv{\Psi}{x}\dd{x}
	\end{bmatrix}=\\
-\Psi\pdv{\phi^*}{x}\eval_{-\infty}^{\infty}\qty(-\frac{\hbar^2}{2m})+
\qty(-\frac{\hbar^2}{2m})\inftyint\pdv[2]{\phi^*}{x}\Psi\dd{x}\\=
\inftyint \qty(-\frac{\hbar^2}{2m}\pdv[2]{x})\Psi^*\Psi \dd{x}
\end{gather*}
Следовательно, $\hat{T}=\hat{T}^+$. Значит оператор кинетической энергии эрмитов.

