%!TEX root = ../quantum.tex
\subsubsection{\textcolor{red}{Операторы физических величин. Какие значения может принимать физическая величина}}

\subsubsection{{Оператор производной физической величины по времени}}

$$\bar A = \mel{\Psi}{\hat A}{\Psi}$$
\textbf{Определение:} $\bar{\dot x}=\dot{\bar x}$, где $\dot x$- оператор производной по времени. 

В общем случае оператор $\hat A$ может зависеть от времени.
Тогда 
$$\dot{\hat A}= 
\mel{\dot \Psi}{\hat A}{\Psi}+ 
\mel{\Psi}{\hat A}{\dot \Psi}+
\mel{\Psi}{\pdv{\hat A}{t}}{\Psi}$$

Когда $\hat A=\hat H$, то нет зависимости от времени оператора, но при этом среднее от времени будет зависеть

$$\dot{\bar x}=\mel{\dot \Psi}{\hat x}{\Psi}+\mel{\Psi}{\hat x}{\dot \Psi} $$

Тогда $$\ket{\dot \Psi}=\frac{1}{i\hbar}\hat H \ket{\Psi}$$
$$\ket{\Psi^+}=\bra{\Psi} $$
$$(\hat H \ket{\Psi})^+=\bra{\Psi}\hat H^+ $$
Тогда 
$$\bra{\dot\Psi} =-\frac{1}{i\hbar}\bra{\Psi} \hat H^+
$$
Учтем, что $\hat H^+=\hat H$



\begin{gather*}
\dot{\bar x}=\frac{i}{\hbar}\qty(\mel{\Psi}{\hat H \hat x}{\Psi}- \mel{\Psi}{ \hat x \hat H}{\Psi})=\frac{i}{\hbar}\mel{\Psi}{\qty(\hat H \hat x- \hat x \hat H)}{\Psi}=
\bar{\hat x}
\end{gather*}
$$\hat{\dot x}=\hat H \hat x- \hat x \hat H=\hat V_x$$

$$H=-\frac{\hbar^2}{2m}\pdv[2]{x}+U(x)$$
$U(x)x-xU(x)=0$- U и x коммутируют.

\begin{gather*}
	\hat V_x=-\frac{i\hbar}{2m}\qty(\pdv[2]{x} x - x\pdv[2]{x})\Psi=-\frac{i\hbar}{2m}
	\qty{x\Psi'' +\Psi'+\Psi'-x\Psi '' }\\=-\frac{i\hbar}{m}\pdv{x}\Psi=\frac{\hat p}{m}\Psi
\end{gather*}
Следовательно $\hat V_x=\frac{\hat p}{x}$

\subsubsection{\textcolor{black}{Задача. $\Psi(x)=\Theta(x+L/2)-\Theta(x-L/2)$. Нормировать, перейти $\Psi(x)\to\Psi(p)$. Найти связь ширины в $x$ и $p$ представлениях}}

Дана ненормированная волновая функция $\Psi(x)$:
\begin{gather*}
\Psi(x)=\Psi(x)=\Theta\left(x+\frac{L}{2}\right)-\Theta\left(x-\frac{L}{2}\right)
\end{gather*}

\textbf{Нормировка}. Предположим, что наша функция есть произведение нормированной $\psi_0$ на произвольную константу $\frac{1}{c}$, найдем ее из условия нормировки:

\begin{gather*}
\int_{-\infty}^{\infty}|\Psi_0(x)|^2 dx=
c^2\int_{-\infty}^{\infty}|\Psi(x)|^2 dx=1\\
c^2\int_{-\infty}^{\infty}|\Psi(x)|^2 dx=
c^2 L \quad\Rightarrow\quad c=\frac{1}{\sqrt{L}}
\end{gather*}
Мы нашли нормированную функцию:

\begin{gather*}
\Psi_0=\frac{1}{\sqrt{L}}\left(
	\Theta\left(x+\frac{L}{2}\right)-\Theta\left(x-\frac{L}{2}\right)
\right)
\end{gather*}

$|\Psi_0|^2$ имеет вид прямоугольника, высота его $\frac{1}{L}$, ширина от $-L/2$ до $L/2$.


\textbf{Переход и связь ширин}. Переход в $p$-предст:
\begin{gather*}
\Psi(p)=\frac{1}{\sqrt{2\pi\hbar}}\int_{-\infty}^{+\infty}\Psi_0(x)e^\frac{-ipx}{\hbar}dx=
\frac{1}{\sqrt{2\pi\hbar L}}\int_{-\infty}^{+\infty}\left(
	\Theta\left(x+\frac{L}{2}\right)-\Theta\left(x-\frac{L}{2}\right)
\right)e^\frac{-ipx}{\hbar}dx=\\=
\frac{1}{\sqrt{2\pi\hbar L}}\int_{-L/2}^{+\infty}e^\frac{-ipx}{\hbar}dx-\frac{1}{\sqrt{2\pi\hbar L}}\int_{L/2}^{+\infty}e^\frac{-ipx}{\hbar}dx=
\frac{1}{\sqrt{2\pi\hbar L}}\int_{-L/2}^{+L/2}e^\frac{-ipx}{\hbar}dx=\\=
\frac{1}{\sqrt{2\pi\hbar L}} \frac{\hbar}{(-ip)} e^\frac{-ipx}{\hbar}\bigg|_{-L/2}^{L/2}=\frac{2\hbar L}{pL\sqrt{2\pi\hbar L}}\sin\frac{pL}{2\hbar}
%
=\frac{L}{\sqrt{2\pi\hbar L}}\,\mathrm{sinc}\,\frac{pL}{2\hbar}
=\sqrt{\frac{L}{h}}\,\mathrm{sinc}\,\frac{pL}{2\hbar}
\end{gather*}

Характерная ширина синка -- это ширина первого лепестка:

$\frac{pL}{2\hbar}=\pi$ -- первый ноль синка, значит ширина лепестка это расстояние между симметричными нулями: $\Delta p = 2p=\frac{2h}{L}$

Учтя, что у нашей вф $\Delta x = L$, окончательно получим:
$$
\Delta x \Delta p = 2h
$$
Верно, соответствует принципу неопределенности с точностью до постоянного множителя.

Выбирая другую характерную ширину синка, больше чем первый лепесток, получим 
$$
\Delta x \Delta p \geq 2h
$$