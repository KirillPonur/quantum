%!TEX root = ../quantum.tex
\subsubsection{Преобразование Фурье как разложение по собственным функциям оператора импульса.}

$-i\hbar \frac{\partial{\Psi(x)}}{\partial x}=p_x \Psi(x)$ - операторное уравнение по собственным функциям импульса.

$-i\hbar \nabla$ - оператор импульса.

$$-i\hbar d\Psi(x)=p_x \Psi(x)dx$$
$$-i\hbar \ln(\Psi(x))=p_xx +C$$
$$\Psi(x)=\widetilde{C} e^{\frac{p_xx}{-i\hbar}}$$
Т.к. $\Psi(x)$ и $\widetilde{C} \Psi(x)$описывают одно и то же состояние квантовой системы, отбросим константу. 

$\Psi(x)=e^{\frac{p_xx}{-i\hbar}}=e^{\frac{ip_xx}{\hbar}}$ - волна Де Бройля, собственная функция оператора импульса. Фурье-образ:
$$\Psi(x)=\int \Psi(p) e^{\frac{ip_xx}{\hbar}} dx$$
 учетом нормировки в одномерном случае:
 $$\Psi(x)=\frac{1}{\sqrt{2\pi \hbar}}\int \Psi(p) e^{\frac{ip_xx}{\hbar}} dx$$

\subsubsection{Докажите, что нормировка сохраняется при замене представления, если использовать нормированные собственные функции }

$$\infint \abs{\Psi(x)}^2 \dd{x} = \infint \abs{\Psi(p)}^2 \dd{p}$$
По теореме Парсиваля:
$$\infint g^*(x) f(x)=\infint G^*(p) F(p) \dd{p}, $$
где $G(p)=F[g(x)]$, $F(p)=F[f(x)]$.

Тогда

$$\infint \Psi(x)^*\Psi(x) \dd{x} = \infint \Psi(p)^*\Psi(p) \dd{p} $$
$$\infint \abs{\Psi(x)}^2 \dd{x} = \infint \abs{\Psi(p)}^2\dd{p}  $$
Что же, осталось только доказать теорему Парсиваля.

{\centering\textbf{Доказательство} \\}

\begin{flushleft}
	
Пусть:

$\phi_0=\phi_1(x)\phi_2(x)$

Известно что:

$\Phi_1(p)=F[\phi_1(x)]$

$\Phi_2(p)=F[\phi_2(x)]$

Найдём:
$\Phi_0(p)=F[\phi_1(x)\cdot\phi_2(x)]$

\end{flushleft}
% Ща снова буит мясо
\begin{gather*}
\Phi_0(p)=\frac{1}{\sqrt{2 \pi \hbar}} 
\infint \phi_1(x)\phi_2(x)\exp{-\frac{ipx}{\hbar}} \dd x =\\
\frac{1}{\sqrt{2 \pi \hbar}}
\infint \frac{1}{\sqrt{2 \pi \hbar}} 
\infint \Phi_1(p')\exp{+\frac{ip' x}{\hbar}}\dd{p'}\cdot \phi_2(x)\exp{-\frac{ipx}{\hbar}}\dd{x}=\\
\frac{1}{2\pi\hbar}
\infint \infint \Phi_1(p')
\exp{\frac{ix(p'-p)}{\hbar}}\dd{p'}
 \cdot\phi_2(x)\dd{x}=\\
\frac{1}{2\pi\hbar}
\infint\infint\phi_2(x)\exp{\frac{ix}{\hbar}(p'-p)}\dd{x}\cdot\Phi_1(p' )\dd{p'}=\\
\qty{\frac{1}{2\pi\hbar}\infint\phi_2(x)\exp{\frac{ix}{\hbar}(p'-p)}=\Phi_2(p-p')}=\\
\frac{1}{\sqrt{2\pi\hbar}}\infint \Phi_2(p-p')\Phi_1(p')\dd{p'}
\end{gather*}

Получили, что 
$$\Phi_0=\frac{1}{\sqrt{2\pi\hbar}}\infint \Phi_2(p-p')\Phi_1(p')\dd{p'}=
\frac{1}{\sqrt{2\pi\hbar}} \infint \phi_1(x)\phi_2(x) e^{-\frac{ipx}{\hbar}} \dd{x}$$

Пусть $p=0$, тогда 
$$\Phi_0(0)=\frac{1}{\sqrt{2\pi\hbar}} \infint \phi_1(x)\phi_2(x)\dd{x}$$
$$\infint \Phi_2(-p' )\Phi_1(p' )\dd{p'}= \infint \phi_1(x) \phi_2(x)\dd{x} $$
По свойству преобразования Фурье:
$$F[\phi^*(x)]=\Phi^*(-p) $$
Получаем:
$$\infint \Phi_2'^*(p)\Phi_1(p') \dd{p'} = \infint \phi_1 \phi^*_2(x) \dd{x} $$
Что и требовалось доказать. \qed

\subsubsection{\textcolor{red} {Выведите уравнения Гейзенберга для частицы в потенциале. Сформулируйте
условие сохранения физической величины. Когда сохраняется энергия? Импульс?} }