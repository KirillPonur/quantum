%!TEX root = ../quantum.tex
\subsubsection{Вопрос 1}
Преобразование Фурье как разложение по собственным функциям оператора импульса.
$-i\hbar \frac{\partial{\Psi(x)}}{\partial x}=p_x \Psi(x)$ - операторное уравнение по собственным функциям импульса.

$-i\hbar \nabla$ - оператор импульса.

$$-i\hbar d\Psi(x)=p_x \Psi(x)dx$$
$$-i\hbar \ln(\Psi(x))=p_xx +C$$
$$\Psi(x)=\widetilde{C} e^{\frac{p_xx}{-i\hbar}}$$
Т.к. $\Psi(x)$ и $\widetilde{C} \Psi(x)$описывают одно и то же состояние квантовой системы, отбросим константу. 

$\Psi(x)=e^{\frac{p_xx}{-i\hbar}}=e^{\frac{ip_xx}{\hbar}}$ - волна Де Бройля, собственная функция оператора импульса. Фурье-образ:
$$\Psi(x)=\int \Psi(p) e^{\frac{ip_xx}{\hbar}} dx$$
 учетом нормировки в одномерном случае:
 $$\Psi(x)=\frac{1}{\sqrt{2\pi \hbar}}\int \Psi(p) e^{\frac{ip_xx}{\hbar}} dx$$
\subsubsection{Вопрос 3}
Докажите, что нормировка сохраняется при замене представления, если использовать нормированные собственные функции  
	