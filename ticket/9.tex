%!TEX root = ../quantum.tex
\subsubsection{Разложите $\delta(x)$  по собственным функциям оператора импульса.}


$\Psi(x)=\frac{1}{\sqrt{2\pi \hbar}}e^{\frac{ip x}{\hbar}}$ - нормированная собственная функция оператора импульса. Разложим по собственным функциям оператора импульса:

$$\delta(x) = \frac{1}{\sqrt{2\pi \hbar}}\int_{-\infty}^{+\infty} C(p) e^{\frac{ip x}{\hbar}} dp$$
$$C(p)=\bra{\Psi^*} \ket{\delta(x)}=\Psi^*(0)$$
$$\Psi^*(0)=\frac{1}{\sqrt{2\pi \hbar}}$$
$$\delta(x) = \frac{1}{\sqrt{2\pi \hbar}}\int_{-\infty}^{+\infty} \frac{1}{\sqrt{2\pi \hbar}} e^{\frac{ip x}{\hbar}} dp$$
$$\delta(x) = \frac{1}{2\pi \hbar}\int_{-\infty}^{+\infty} \ e^{\frac{ip x}{\hbar}} dp$$

\subsubsection{\textcolor{red} {Построение оператора эволюции путем разложения по собственным функциям стационарного уравнения.} }

\subsubsection{{Напишите оператор координаты в импульсном представлении.} }

Этот вопрос будет рассматриваться позднее в билете 20. Ниже фрагмент двадцатого билета:

\begin{equation}
	\label{eq:20.1a}
	\mel{a}{\hat{L}}{\tilde a}=\int \dd{b}\dd{\tilde b} \Psi_a^*(b)\mel{b}{\hat L}{\tilde b} \Psi_{\tilde a}(\tilde b)
\end{equation}

Проиллюстрируем формулу \eqref{eq:20.1a} на примере нахождения явного вида оператора $\hat x$ в p-представлении по известному виду этого же оператора в x-представлении. В x-представлении $\mel{x}{\hat x}{\tilde x}=x\delta(x-x').$ Из \eqref{eq:20.1a} получаем:
\begin{gather*}
	\mel{p}{\hat x}{\tilde p}=\int \dd x \dd{\tilde x} \Psi_p^(x)\mel{x}{\hat x }{\tilde x}\Psi_{\tilde p}(\tilde x)=
	\\
	\frac{1}{2\pi\hbar}\int \dd x \dd{\tilde x} e^{-ipx/\hbar}e^{i\tilde p \tilde x/\hbar}x \delta(x-x')=\frac{1}{2\pi\hbar}\int x\dd x e^{-ix(p-\tilde p)/\hbar}=\\
	i\hbar\pdv{p}\int \frac{\dd{y}}{2\pi}e^{-y(p-\tilde p)}=i\hbar\pdv{p} \delta(x-x')
\end{gather*}