%!TEX root = ../quantum.tex
\subsubsection{Покажите, что если два оператора, A и B имеют общие собственные функции, то
они коммутируют.}


\begin{gather*}
	\hat{B}\cdot\eval \hat{A}\Psi_n=a_n\Psi_n \\
	\hat{A}\cdot\eval \hat{b}\Psi_n=b_n\Psi_n
\end{gather*}
\begin{equation*}
\begin{cases}
	\hat{B}\hat{A}\Psi_n=a_n\hat{B}\Psi_n \\
	\hat{A}\hat{B}\Psi_n=b_n\hat{A}\Psi_n
\end{cases}
\end{equation*}

\begin{equation*}
\begin{cases}
	\hat{B}\hat{A}\Psi_n=a_nb_n\Psi_n \\
	\hat{A}\hat{B}\Psi_n=b_na_n\Psi_n \\
\end{cases}
\end{equation*}
Вычтем из первого уравнения второе:
\begin{equation*}
	(\hat{B}\hat{A}-\hat{A}\hat{B})\Psi=\sum\limits_n a_n (\hat{B}\hat{A}-\hat{A}\hat{B})\Psi_n=0
\end{equation*}
Следовательно $\hat{A}$  и $\hat{B}$ коммутируют.

\subsubsection{\textcolor{red}{Как выглядит оператор в своём собственном представлении. Как устроены волновые функции в этом представлении?}}

\subsubsection{\textcolor{red}{Напишите уравнение движения квантовой частицы в однородном поле $U=-Fx$ в импульсном представлении}}