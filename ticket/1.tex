%!TEX root = ../quantum.tex

\subsubsection{Понятие состояния}\hypertarget{section}{}\label{section}

Квантовая система в силу своей малости не подчиняется законам классич. физики. В классике состояние системы -- это набор параметров, которые полностью описывают эволюцию системы.

Однако, в квантовой физике будущее не детерминировано, так как координаты и импульсы не могут быть определены в каждый момент времени одновременно. Если квантово-механическая система в настоящий момент времени определена наиболее полновозможным образом, то поведение системы в следующий момент времени принципиально неоднозначно. Состояние квантово-механической системы - это набор параметров, которые дают наиболее полную информацию о квантово-механической системе. 
% <!-- -->

В нотации Дирака вектор состояния будет $|\Psi>$ (абстрактный вектор, не привязанный к системе координат). 
% <!-- -->


\subsubsection{Волновая функция и её физический смысл}\hypertarget{section-1}{}\label{section-1}

% <!-- -->
Волновая функция $\Psi$ - это комплексная функция, которая описывает состояние квантово-механической системы, и является коэффициентом разложения квантового состояния по базису. Если базис координатный, то это функция $\Psi(x,t)$, аргументами которой являются координаты (м.б. обобщенные), так называемое \guillemotleft{}$x$-представление\guillemotright{}. Аналогично есть импульсное $\Psi(p,t)$ $p$-представление. 
% <!-- -->

Физ.смысл имеет $|\Psi|^2$ -- плотность вероятности. 
% <!-- -->


\subsubsection{Как вычислить распределение вероятности какой либо физической величин}\hypertarget{section-2}{}\label{section-2}

% <!-- -->
$|\Psi(x,t)|^2$ есть плотность вероятности нахождения частицы в координате x (в одномерном пространстве), $|\Psi(p,t)|^2$  плотность вероятности нахождения импульса частицы в p. Каждой наблюдаемой (физической) величине в квантовой физике соответствует свое представление волновой функции. 
% <!-- -->

\subsubsection{\textcolor{red}{Преобразования операторов и векторов состояний. Унитарные операторы – операторы сохраняющие ортонормированность.}}


\subsubsection{Задача (отнормировать/нарисовать $\psi(x)=({1+e^{i\pi/4}})/{\sqrt{x^2+a^2}}$)}\hypertarget{section-3}{}\label{section-3}

% <!-- -->
Дана ненормированная волновая функция $\psi(x)$:

\begin{displaymath}
\psi(x)=\frac{1+e^{i\pi/4}}{\sqrt{x^2+a^2}}
\end{displaymath}

Отнормируйте и нарисуйте график плотности вероятности величины $x$.
% <!-- -->

\textbf{Решение}. Множитель без \guillemotleft{}$x$\guillemotright{} может быть отброшен (так как домножение на комплексную константу не изменяет ненормированную волновую функцию). 
% <!-- -->

Тогда задача сводится к нормировке следующей функции:

\begin{displaymath}
\psi'(x)=\frac{1}{\sqrt{x^2+a^2}},\qquad \psi_0=c\psi(x)
\end{displaymath}

Предположим, что наша функция есть произведение нормированной $\psi_0$ на произвольную константу $c$, найдем ее из условия нормировки:

\begin{displaymath}
\int_{-\infty}^{\infty}|\psi_0(x)|^2 dx=
c^2\int_{-\infty}^{\infty}|\psi'(x)|^2 dx=1
\end{displaymath}

\begin{displaymath}
\int_{-\infty}^{\infty}|\psi'(x)|^2 dx=
\int_{-\infty}^{\infty} \frac{1}{x^2+a^2} dx= \frac{1}{a}\atan\frac{x}{a}\bigg|_{-\infty}^{\infty}=
\frac{1}{a}(\frac\pi2+\frac\pi2)=\frac{\pi}{a}
\end{displaymath}

Тогда  $c^2=\frac{1}{\pi/a}=\frac{a}{\pi} \Rightarrow c=\sqrt\frac{a}{\pi}$.

Мы нашли нормированную функцию:

\begin{displaymath}
\psi_0=\sqrt\frac{a}{\pi}\frac{1}{\sqrt{a^2+x^2}},\qquad
|\psi_0|^2=\frac{a}{\pi(a^2+x^2)}
\end{displaymath}

% <!---->
Колокообразный вид функции $|\psi_0|^2$ легко нарисовать,  если учесть, что 1) она всегда положительна 2) имеет максимум там, где корень имеет минимум (т.е. в точке $x=0$, тогда максимум равен $\frac{1}{\pi a}$).

