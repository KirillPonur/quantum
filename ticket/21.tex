%!TEX root = ../quantum.tex
\subsubsection{Вычислите оператор, сопряженный к произведению AB . Сформулируйте условие эрмитовости произведения, если $A$ и $B$ эрмитовы.}



Оператор $A^*$ называется сопряженным к оператору $A$ если $(\phi, Ax)=(A^*\phi,x).$

Вычислим оператор, сопряженный к произведению $AB$.

$$(\phi, ABx)= (A^*\phi,Bx)*=(B^*A^*\phi,x) $$
(*) к $A^*\phi$  относимся как к целому.

Тогда $(AB)^*=B^*A^*.$

Условие эрмитовости произведения, если $A$ и  $B$ эрмитовы:
произведение двух эрмитовых операторов является эрмитовым, если их коммутатор равен 0.

\subsubsection{Докажите, что собственные функции эрмитового оператора, соответствующие разным собственным значениям, ортогональны.}


\begin{equation}
\label{eq:3.1}
A\ket{\Psi_n}=a_n\ket{\Psi_n} 	
\end{equation}

Берем эрмитовое сопряжение:
\begin{equation}
	\label{eq:3.2}
\bra{\Psi_n}A^+=a_n^*\bra{\Psi_n} 
\end{equation}
Заменим в \eqref{eq:3.2} $n\rightarrow m$. Уравнение \eqref{eq:3.1} скалярно умножим на $\bra{\psi_m}$, \eqref{eq:3.2} умножим на $\ket{\Psi_n}$  и вычтем.

Получаем:
\begin{equation}
	\bra{\Psi_m}A\ket{\Psi_n}-\bra{\Psi_n}A^+\ket{\Psi_m}=(a_n-a^*_m)
	\braket{\Psi_m}{\Psi_n}=0
\end{equation}
n и m - разные, значит $a_n\neq a_m^*$. Следовательно $\braket{\Psi_m}{\Psi_n}=0$

$$\big \braket{\Psi_m}{\Psi_n}=\int \Psi_m^*(x)\Psi_n(x)\dd{x}=(\Psi_m,\Psi_n)=0 $$