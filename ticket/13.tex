%!TEX root = ../quantum.tex
\subsubsection{Нахождения волновой функции по данным одного из представлений}

$$\Psi(x)=\exp \qty{-\frac{(x-x')^2}{4b^2}+iUx} $$
В этом билете функция нормироваться не будет, так как автору стало лень, а на самом деле её и не просят. К тому же, на полу ширину нормировка не влияет.

$$\Psi(p)=\frac{1}{\sqrt{2\pi\hbar} } \infint \Psi(x) \exp{-\frac{ipx}{\hbar}}\dd{x} $$
Иначе говоря, $\Psi(p)=F[\Psi(x)]$.

По свойствам преобразования Фурье:  
\begin{enumerate}
	\item $\Psi_1(p-U\hbar)= F[\Psi_1(x)e^{iUx}]$ $\Psi_1(x)=\exp \qty{-\frac{(x-x')^2}{4b^2}}$
	\item $\Psi_2(p)\exp{-\frac{ipx'}{\hbar}}=F[\Psi_2(x-x')]$, где 
	$\Psi_2(x)=\exp \qty{-\frac{(x)^2}{4b^2}}$
\end{enumerate}

Найдем преобразование Фурье $\Psi_2(x)$ по определению.

\begin{gather*}
\Psi_2(p)=\frac{1}{\sqrt{2\pi\hbar}}\infint \exp{-\frac{x^2}{4b^2}+\frac{-ipx}{\hbar}}=\\
\frac{1}{\sqrt{2\pi\hbar}}\infint \exp{-\qty(\frac{x}{2b}+\frac{ipb}{\hbar})^2+
\qty(\frac{ipb}{\hbar})^2}\dd{x}= \\
\frac{1}{\sqrt{2\pi\hbar}}\exp{-\frac{p^2b^2}{\hbar^2}} 
\infint \exp{-\qty(\frac{x}{2b}+\frac{ipb}{\hbar})^2} \dd{x}=\\
\frac{2b}{\sqrt{2\pi\hbar}}\exp{-\frac{p^2b^2}{\hbar^2}} 
\infint \exp{-y^2}\dd{y}=\frac{2b\sqrt\pi}{\sqrt{2\pi b}}\exp{-\frac{p^2b^2}{\hbar^2}}=\\ 
\sqrt{\frac{2}{\hbar}}b\exp{-\frac{p^2b^2}{\hbar^2}}
\end{gather*}

Получим, что $\Psi_2(p)=\sqrt{\frac{2}{\hbar}}b\exp{-\frac{p^2b^2}{\hbar^2}}$.

$$\Psi_2(p)=F[\exp{-\frac{x^2}{4b^2}}]$$

Согласно свойству (2):
$$\sqrt{\frac{2}{\hbar}}b\exp{-\frac{p^2b^2}{\hbar^2}}\exp{-\frac{px'}{\hbar}}=
F[\Psi_2(x-x')]=F\qty[ \exp{-\frac{(x-x')^2}{4b^2}} ] = F[\Psi_1(x)]$$
Согласно свойству (1):
\begin{gather*}
\sqrt{\frac{2}{\hbar}}b\exp{-\frac{(p-U\hbar)^2b^2}{\hbar^2}}\exp{-\frac{i(p-U\hbar)x'}{\hbar}}=\\
 F\qty[\exp{-\frac{(x-x')^2}{4b^2}}+iUx]=F[\Psi(x)]
\end{gather*}
Что и требовалось найти.

\begin{equation}
	\Psi(p)=\sqrt{\frac{2}{\hbar}}b\exp{-\frac{(p-U\hbar)^2b^2}{\hbar^2}-\frac{i(p-U\hbar)x'}{\hbar} }
\end{equation}

Теперь найдем $\Delta p$. Она в $\Psi(p)$ такая же, как в $\Psi_2(p)$. Ищем полуширину на уровне $\frac{1}{e}$. Тогда
$$\Delta p=\frac{2\hbar}{b}$$
Найдем $\Delta x$на уровне $\frac1e$, при этом отбрасывая фазу.

\begin{gather*}
	-\frac{(x-x')^2}{4b^2}=-1 \\
	x_1=2b+x'\\
	x_2=x'-2b\\
	\Delta x= x_2-x_1=4b
\end{gather*}

$$\Delta x\cdot \Delta p=\frac{4b\cdot2\hbar}{b}=8\hbar  $$

\subsubsection{\textcolor{red} {На примере конкретного пакета продемонстрируйте соотношение неопределенности. Операторы как матрицы. Непрерывные и дискретные индексы.} }


\subsubsection{\textcolor{red} {Представления операторов умножения и дифференцирования как матриц.} }