%!TEX root = ../quantum.tex

\subsubsection{{Вывести соотношения действия/гамильтониана/импульса ($\nabla S=\vec{p},\frac{\partial S}{\partial t}=-H$)}}

$$ S[q,t]\definition \int \limits_t^{t_{f}} L(q,\dot q,t )\dd{t}$$
L-функция Лагранжа.

\begin{gather*}
	\delta S= \int \qty[\pdv{L}{q}\cdot \delta q + \pdv{L}{q} \delta \dot q]\dd{t}=\\
	\qty{\dv{t}\qty(\frac{L}{\dot q} \delta q) = \pdv{L}{q}\delta \dot q +
	\delta \dv{t}\qty(\pdv{L}{\dot q})
	}
	\\
	\qty{
	\pdv{L}{q} \delta \dot q= \dv{t}\qty(\pdv{L}{\dot q} \delta q)- \delta q
	\dv{t}\qty(\pdv{L}{\dot q})
	}\\
	=\int \qty[\pdv{L}{q}-\dv{t}\qty(\pdv{L}{\dot q})]\delta q \dd t + \pdv{L}{\dot q}\delta q \eval_t^{t_f}= \\
	\int \qty[\pdv{L}{q}-\dv{t}\qty(\pdv{L}{\dot q})]\delta q \dd t=0
\end{gather*}
Значит 
$$\pdv{L}{q}-\dv{t}\qty(\pdv{L}{\dot q})=0 $$
$$L=p\cdot \dot q -H, ~ p\definition\pdv{L}{\dot q}$$
Рассмотрим $S[q,t]$ не как функционал а как функцию верхнего и нижнего предела, то есть $S[q_i,t_i'q_f,t_f]$

$$\delta S= p \delta q + 	\int \qty[\pdv{L}{q}-\dv{t}\qty(\pdv{L}{\dot q})] \delta q \dd t  $$
$$p=\frac{\delta S}{\delta q}$$

Если обобщенных координат много, то $\pdv{S}{q_i}=p_i$. Значит $\nabla S=\vec p$.

\begin{gather*}
	\begin{cases}
		\dv{S}{t}=L=p\cdot \dot q -H \\
		\dv{S}{t}=\pdv{S}{q} \dot q +\frac{S}{t}
	\end{cases}
	\eval \Longrightarrow \pdv{S}{t}=-H
\end{gather*}


\subsubsection{{Чему равно среднее физической величины $A$. Определение через волновую функцию в $x$ -- представлении, некотором $B$ представлении, и $A$ представлении}}
\textbf{Вариант Ильи}:


$$\mean{A}=\infint \Psi^*(x)\hat A \Psi(x) \dd{x}- \text{ квантово-механическое усреднение}$$
Через произвольные функции в $x$- представлении

\begin{gather*}
\mean{A}=\infint A \abs{\Psi(A)}^2 \dd{A}- \text{ вероятность того, что система принимает состояние} \Psi(A) \\ \text{ в промежутке} \dd{A} 
\end{gather*}

$$\mean{A}=\infint \Psi^*(B) \hat A \Psi(B) \dd{B}- \text{среднее в некотором представлении }, $$
$$ \hat A- \text{оператор в B представлении}$$

\textbf{Вариант Ани}:

По определению среднего:
$$\bar A= \sum\limits_n a_n\abs{C_n}^2 =\sum\limits_n a_n C_n C_n^*$$
\begin{gather*}
	\bar A= \sum\limits_n a_n \int \Psi(x' )\Psi_n^*(x' )\dd{x'}\int \Psi^*(x' )\Psi_n(x' )\dd{x' }=
	\\
	\int \Psi^*(x' )\sum\limits_n a_n \Psi_n^*(x' )\Psi_n(x')\Psi(x' )\dd{x} \dd{x' }=
	\\
	\qty{K(x,x')=\sum\limits_n a_n\Psi_n^*(x'')\Psi(x' )-\text{ядро оператора}}
	\\=
\int	\Psi^*(x')K(x,x')\Psi(x' )\dd{x}\dd{x' }=\int\Psi^*(x)\hat A \Psi(x) \dd{x}
\end{gather*}
$$\hat A \Psi(x)=\int K(x,x' )\Psi(x' )\dd{x' }$$

$$\bar{\hat A}= \mel{\Psi}{\hat A}{\Psi}$$--
определение через волновую функцию в x- представлении.

Пусть есть оператор $\hat B$

$$\hat B \phi_n(x)=b_n \phi_n(x)$$

$$\phi(x)=\sum\limits_n C_n \phi_n (x)$$
$$\phi^*(x)=\sum\limits_m C_m^* \phi_m^*(x)$$

$$\bar A=\int \sum\limits_{n,m} C_m^*C_n (\phi_m^* \hat A \phi_n) \dd{x} = \sum\limits_{n,m} C_m^* C_n A_{mn}$$

$$\bar{\hat A} = \sum\limits_{n,m} C_m^* C_n A_{mn}$$
-- определение через B-представление

$$\hat A \Psi_n(x)=a_n \Psi_n(x) $$
Скалярно умножим на $\Psi_m^*(x)$

$$\mel{\Psi_m}{\hat A}{\Psi_n}=a_m \braket{\Psi_m}{\Psi_n}=a_m \delta_{mn}=A_{mn}$$
$$\bar A=A_{mn}$$-- в собственном представлении.

\subsubsection{{Задача. Эксп. регуляризацией доказать $\frac{1}{2\pi}\int\exp(-k(x-x'))dk=\delta(x-x')$ + на языке Дирака + две интерп.}}

$$\infint e^{ik(x-x')}\frac{\dd{k}}{2\pi}=\delta (x-x' )$$

\begin{gather*}
\lim\limits_{\lambda\rightarrow 0} \infint e^ {ik\xi+\lambda \abs{k}} \dd{k}=
\lim\limits_{\lambda\rightarrow 0} 
	\qty(\int \limits_{-\infty}^0 e^ {ik\xi-\lambda k} \dd{k}
+\lim\limits_{\lambda\rightarrow 0} 
	\int \limits_{0}^{\infty} e^ {ik\xi+\lambda k} \dd{k})
=\\\lim\limits_{\lambda\rightarrow 0} \qty[\frac{1}{i\xi-\lambda}\cdot\exp{ik\xi- \lambda k	 }\eval\limits_0^{\infty}
+\frac{1}{i\xi+\lambda}\cdot\exp{ik\xi- \lambda k	 }\eval\limits^0_{-\infty}] =\\
\lim\limits_{\lambda\rightarrow 0} \qty[-\frac{1}{i\xi- \lambda}+\frac{1}{i\xi+ \lambda}]= \lim\limits_{\lambda\rightarrow 0} \frac{-i\xi- \lambda - i \xi - \lambda}{- \xi^2 - \lambda^2}= 
\\
\lim\limits_{\lambda\rightarrow 0} \frac{2\lambda}{\xi^2+ \lambda^2}
\equiv  \lim\limits_{\lambda\rightarrow 0} f_{\lambda}(\xi)
\end{gather*}