%!TEX root = ../quantum.tex

\subsubsection{\textcolor{red}{Вывести соотношения действия/гамильтониана/импульса ($\nabla S=\vec{p},\frac{\partial S}{\partial t}=-H$)}}

\subsubsection{\textcolor{red}{Чему равно среднее физической величины $A$. Определение через волновую функцию в $x$ -- представлении, некотором $B$ представлении, и $A$ представлении}}

$$\mean{A}=\infint \Psi^*(x)\hat A \Psi(x) \dd{x}- \text{ квантово-механическое усреднение}$$
Через произвольные функции в $x$- представлении

\begin{gather*}
\mean{A}=\infint A \abs{\Psi(A)}^2 \dd{A}- \text{ вероятность того, что система принимает состояние} \Psi(A) \\ \text{ в промежутке} \dd{A} 
\end{gather*}

$$\mean{A}=\infint \Psi^*(B) \hat A \Psi(B) \dd{B}- \text{среднее в некотором представлении }, $$
$$ \hat A- \text{оператор в B представлении}$$

\subsubsection{\textcolor{red}{Задача. Эксп. регуляризацией доказать $\frac{1}{2\pi}\int\exp(-k(x-x'))dk=\delta(x-x')$ + на языке Дирака + две интерп.}}