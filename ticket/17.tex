%!TEX root = ../quantum.tex
\subsubsection{Вопрос 1}

% \newcommand{\definition}{\underset{def}{=}}
% \renewcommand{\big}{\displaystyle}

Дайте определение сопряженного по Эрмиту оператора. Ответ сформулируйте в явной интегральной форме в $x$ представлении, в обозначениях Дирака и обычных абстрактных векторных обозначениях.

Сопряженным по Эрмиту оператором называется оператор, который и транспонирован, и сопряжен
$$A^+=A^{*T} $$
Сопряженным по Эрмиту оператором называется такой оператор, что 
$(\phi,A\Psi)\definition(A^+\phi,\Psi)$

\textbf{В интегральной форме:}

$$\big \int\phi^*(x)K(x,x')\Psi(x')\dd{x}\dd{x'}\definition\int N^*(x,x')\phi(x')\dd{x'}\Psi(x)\dd{x}=*$$
Обозначим ядро оператора $A^+$ как $N(x,x')$. Тогда $A^+\phi=\int N(x,x)'\phi(x')\dd{x'}$

$$\big *=\int N^*(x',x)\phi^*(x)\dd{x}\Psi(x')\dd{x'}$$