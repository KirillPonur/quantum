%!TEX root = ../quantum.tex
\subsubsection{Оператор импульса. Связь с оператором сдвига.}



Оператор сдвига:
$$\hat{T}f(x)=f(x+a) $$

Пусть сдвиг мал $\delta a$.
$$\hat{T_{\delta a}}f(x)=f(x+\delta a)$$

Разложим в ряд Тейлора:
$$f(x+\delta a)= f(x) + \pdv{f}{x} \delta a + \pdv[2]{f}{x} (\delta a)^2+\dots$$
Домножим на $-i\hbar$ и поделим на $-i\hbar$:
$$p_x=-i\hbar\pdv{x}$$
$$f(x+\delta a)=f(x)+\qty(-\frac{\delta a}{i \hbar})pf+\qty(-\frac{\delta a}{i \hbar})^2p^2f+\dots $$
Получаем:
$$\hat{T}f(x)=\exp\qty(-\frac{\delta a}{i \hbar} p)f(x) - \text{связь оператора сдвига и импульса}$$


\subsubsection{\textcolor{red} {Замена представления. Обозначения Дирака.} }


\subsubsection{Запишите нестационарное уравнение Шредингера в энергетическом представлении.
Найдите его общее решение. Как выглядит оператор H
в энергетическом
представлении}

Запишем уравнение Шредингера ($ \dot{\psi} \equiv \frac{\partial \varphi}{\partial t} $)
$$i\hbar \dot{\psi} = \hat{H}\psi $$
Представим $\psi$ в виде разложения по собственным функциям: 
$$ \psi(t,x) = \sum_n C_n(t)\varphi_n(x) $$ 

Тогда $$ \sum_n i\hbar \dot{C_n(t)}\varphi_n(x) - C_n\hat{H} \varphi_n(x) = 0 $$ 

$$  \hat{H}\varphi_n(x) = E_n \varphi_n(x) $$

$$  \sum_n  (i\hbar \dot{C_n(t)} - C_n E_n) \varphi_n(x) = 0   $$

Слева скалярно домножим на $ \varphi_m(x)$:
$$  \bra{\varphi_m} \sum_n  (i\hbar \dot{C_n(t)} - C_n(t) E_n) \ket{ \varphi_n(x)} =0 $$
$$ \bra{\varphi_m}\ket{ \varphi_n} = \delta_{mn} =>  i\hbar \dot{C_m(t)} - C_m(t) E_m=0$$
Полученное уравнение - уравнение Шредингера в энергетическом представлении 
$$ C_m = C_0 e^{-\frac{iE_m}{\hbar}t} $$
$$ => \psi(x,t) = \sum_n C_0 e^{-\frac{iE_m}{\hbar}t} \varphi_n(x) \text{ - общее решение УШ} $$
Домножим УШ скалярно слева на $ \bra{x}$:
$$ i\hbar\bra{x}\ket{\dot{\psi(t)}} = \hat{H} \bra{x}\ket{\psi(t)}$$
$$ i\hbar\bra{x}\hat{1}\ket{\dot{\psi(t)}} = \hat{H} \bra{x}\hat{1}\ket{\psi(t)}$$
$$\hat{1} = \sum_n \ket{n}\bra{n} = \sum_n \ket{\varphi_n} \bra{\varphi_n} \text{ - подставим в уравнение выше}$$ 
$$\sum_n i\hbar \bra{x}\ket{n}\bra{n}\ket{\dot{\psi(t)}} = \hat{H}\sum_n\bra{x}\ket{n}\bra{n}\ket{\psi(t)}$$
Скалярно умножим слева на $\bra{m}\ket{x}$ и возьмем $ \int dx$(???). Тогда $i\hbar \bra{n}\ket{\psi(t)} = \sum_n \bra{m}
\hat{H} \ket{n}\bra{n}\ket{\psi(t)}$, $\hat{H}$ стала диагональной
$$ \hat{H}\ket{n} = E_n \ket{n} \text{ - домножим скалярно на } \bra{m}$$
$ \bra{m} \hat{H} \ket{n} =  E_n \bra{m}\ket{n} =E_n \delta_{mn} $ - диагональная матрица. На диагонали стоят $E_n$(Это
в энергетическом представлении)  
