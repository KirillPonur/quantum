%!TEX root = ../quantum.tex
\input{text/diss}

\begin{document}

\begin{center}
	Билет 7	
\end{center}
3. Найдите $\Psi(p)$ по данным $\Psi(x)$. Найдите связь ширины в $x$ и $p$ представлениях. Сначала найдите нормировочный множитель $\Psi(x)=\exp{-u|x|}$

% \begin{wrapfigure}{l}{0.3\linewidth}
% \includegraphics[width=\linewidth]{fig/fig73}
% \caption{}
% \vspace{-17pt}
% \end{wrapfigure}

$\Psi_0(x)=C\Psi(x)$ - нормировочная функция.

$$\int_{-\infty}^{+\infty}|\Psi_0(x)|dx = \int_{-\infty}^{+\infty}|\Psi(x)|^2C^2dx=1$$

$$\int_{-\infty}^{+\infty} e^{-2u|x|}dx=\int_{-\infty}^{0} e^{2ux}dx+\int_{0}^{+\infty} e^{-2ux}dx=\frac{1}{u}$$
$$\frac{C^2}{u}=1 \Longrightarrow C=\sqrt{u}$$
$$\Psi_0(x)=\sqrt{u}\exp{-u|x|}$$
$$\Psi(p)=\frac{1}{\sqrt{2\pi \hbar}}\int_{-\infty}^{+\infty} \Psi_0(x) e^{\frac{-ipx}{\hbar}}dx=\frac{\sqrt{u}}{\sqrt{2\pi \hbar}}\int_{-\infty}^{+\infty} e^{-u|x|} e^{\frac{-ipx}{\hbar}}dx=\sqrt{\frac{u}{2\pi \hbar}}\int_{-\infty}^{+\infty} e^{-(u|x|+\frac{ipx}{\hbar})}dx$$

Когда х>0
$$\int_{0}^{+\infty} e^{-(ux+\frac{-ipx}{\hbar})}dx=\int_{0}^{+\infty} e^{-x(u+\frac{ip}{\hbar})}dx=-\frac{1}{u+\frac{ip}{\hbar}}e^{-x(u+\frac{ip}{\hbar})} \bigg|_0^{+\infty}=\frac{\hbar}{\hbar u+ip}$$

Когда х<0
$$\int_{-\infty}^{0} e^{-(-ux+\frac{-ipx}{\hbar})}dx=\int_{-\infty}^{0} e^{-x(-u+\frac{ip}{\hbar})}dx=-\frac{1}{-u+\frac{ip}{\hbar}}e^{-x(-u+\frac{ip}{\hbar})} \bigg|_{-\infty}^0=-\frac{\hbar}{-\hbar u+ip}$$

$$\frac{\hbar}{\hbar u+ip}-\frac{\hbar}{-\hbar u+ip}=...=\frac{2\hbar ^2u}{p^2+\hbar ^2u^2}$$


$$\Psi(p)=\sqrt{\frac{u}{2\pi \hbar}} \cdot \frac{2\hbar ^2u}{p^2+\hbar ^2u^2}$$

% \begin{wrapfigure}{l}{0.3\linewidth}
% \includegraphics[width=\linewidth]{fig/fig731}
% \caption{}
% \vspace{-17pt}
% \end{wrapfigure}

 Ширина находится на уровне половины от максимума.
 $$\Psi(p=0)=\sqrt{\frac{2}{u\pi \hbar}}$$

 $$\Psi(p)=\sqrt{\frac{u}{2\pi \hbar}} \cdot \frac{2\hbar ^2u}{(1+\frac{p^2}{\hbar ^2u^2})\hbar ^2u^2}=\sqrt{\frac{1}{2\pi \hbar u}} \cdot \frac{2}{(1+\frac{p^2}{\hbar ^2u^2})}=\sqrt{\frac{2}{u\pi \hbar}} \cdot \frac{1}{(1+\frac{p^2}{\hbar ^2u^2})}=\Psi(p=0)\cdot \frac{1}{(1+\frac{p^2}{\hbar ^2u^2})}$$

 $$\Psi(p^*)=\frac{\Psi(p=0)}{2}$$
 $$\frac{1}{(1+\frac{p^{*2}}{\hbar ^2u^2})}=\frac12 \Longrightarrow p^*=\pm \hbar u $$ 
 $$\Delta p=2\hbar u$$

Найдем $\Delta x$ на уровне $e^{-1}$

$$\Psi(x^*)=\frac{\Psi(p=0)}{e} \Longrightarrow \exp{-u|x^*|}=\exp{-1} \Longrightarrow x^*=\pm \frac{1}{u}$$
 $$\Delta x=\frac{2}{u}$$
 $$\Delta x \cdot \Delta p= 4\hbar$$
 Соотношение неопределенностей позволяет проверить правильность решения. Произведение должно быть равно $\hbar$ с точностью до числового множителя.
\end{document}